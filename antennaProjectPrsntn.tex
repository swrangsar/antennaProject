\documentclass[mathserif]{beamer}
\usepackage{graphicx}
\usetheme{Frankfurt}
\useinnertheme{rounded}

\title{Shorting pin loaded dual-band compact rectangular microstrip antenna}
\author{Chaitanya Pande and
Swrangsar Basumatary}
\institute{IIT Bombay, Powai}
\date{November 27, 2013}

    
\begin{document}
    \frame{\titlepage}
    
    \begin{frame}{LEDs vs Laser Diodes for short-range communication}
	    \pause
	    \textbf{Why use LEDs when Laser Diodes are faster?}
            \pause For broadband short-range optical fiber communications, like
LANs and Fiber-in-the-Home networks, LEDs are:
            \pause
            \begin{itemize}[<+->]
                \item cheaper
                \item safer for the human eyes
                \item less sensitive to temperature variations
                \item and more durable
            \end{itemize}
  
        \pause
        \textbf{Disadvantage of using LEDs}
            \begin{itemize}
                \pause \item The problem with LED is \pause \emph{low modulation rate!}\\
                \pause \item While laser diodes have reached to tens of Gbps, \pause 
                commercial DH-LED (double heterostructure) is still limited at 100 Mbps.
            \end{itemize}
        
    \end{frame}
    
    \begin{frame}{Efforts that were not successful}
        \pause
        Efforts have been made to get upto 500 Mbps for conventional LEDs using
        \begin{itemize}
            \pause \item multilevel Pulse Amplitude Modulation (M-PAM)
            \pause \item and discrete multitone modulation (DMT) \\~\\
        \end{itemize} 
        
        \pause But these techniques are \emph{highly complex} compared to the simple on-off keying (OOK) direct modulation scheme.
    \end{frame}
    
    
    
    \begin{frame}{Limitations of OOK modulation rate for LED}
        \pause
        OOK direct modulation rate of LED is limited by two factors:
        
        \begin{enumerate}
            
            \pause \item spontaneous carrier recombination lifetime
            \begin{itemize}
                \pause \item depends on the material and thus it's beyond our control
                \pause \item typically in the range of a few nanoseconds
            \end{itemize}
            
            \pause \item response time (rise time + decay time) of the LED  
            \begin{itemize}
                \pause \item depends on capacitance and dynamic resistance of p-n junctions
                \pause \item \emph{but this can be controlled!} \\~\\
            \end{itemize}
            
        \end{enumerate}
        
        \pause
        So, the only way to increase OOK modulation rate of LED is by \emph{reducing the rise and decay times of the \textbf{EL} signal.}       
    \end{frame}


   
    \begin{frame}{Ways to reduce the response time}
        \pause
        We look at \emph{two} promising ways of improving the response time of the LED have been found upto now
        \begin{enumerate}
            \pause \item using highly-doped InGaAsP/InP Surface Emitting LED with high current density
            \pause \item using Novel LED's driver circuit \\~\\
        \end{enumerate}
     \end{frame}
    
    
    \begin{frame}{Response of an InGaAsP/InP SE LED}
        \pause
        In an undoped InGaAsP/InP Surface Emitting LED
        \begin{itemize}
            \pause \item the rise time decreases with increasing current
            \pause \item but the decay time is higher than the rise time and independent of current
        \end{itemize}
        \pause
 	    \begin{figure}
             \centering
%             \includegraphics[height=0.4\textheight]{undopedSCLEDResponse.png}
        \end{figure}
        \pause
        But with \emph{heavy zinc-doping concentration} in the active layer and low impedance driving circuit the fall time decreases.
    \end{frame}
    
    

    
    \begin{frame}{References}
        \pause
        \begin{itemize}
                 \item P. H. Binh, V. D. Trong, C.T. Anh, P. Renucci, X. Marie, C. T. Truong, A. T. Pham, ``Novel LED's Driver Circuit for Broadband Short-Range Optical Fiber Communication Systems,'' \emph{Communications and Electronics (ICCE), 2012 Fourth International Conference on}.
                 \item Akira Suzuki, Toshio Uji, Yasumasa Inomoto, Junji Hayashi, Yoichi Isoda, and Hidenori Nomura, ``InGaAsP/InP 1.3-$\mu$m Wavelength Surface-Emitting LED's for High-Speed Short-Haul Optical Communication Systems,'' \emph{IEEE Transactions on Electron Devices}, December 1985.
        \end{itemize}
    \end{frame}
    
    
\end{document}
